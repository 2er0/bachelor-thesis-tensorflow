\chapter{Begriffe im Maschinellen Lernen}
\label{cha:Begriffe}

Diese Erklärung der Begriffe und Elemente verfolgt zwei Ziele.
Zum einen stellt die Grundlage des gesamten Themas dar und soll für Interessierte die nicht so vertraut sind, eine Einführung in die Thematik bieten. Des weiteren werden viele dieser Begriffe noch häufig zum Einsatz kommen (u.A. Neuron, Aktivierungsfunktion, …).


\section{Data Science}



\section{Maschine Intelligenz}

\section{Machine Learning}

\section{Neuronale Netzwerke}

\section{Neuron}

\section{Ebenen/Layer}

\section{Informationen Merken}

\section{Allgemeine Probleme}

\subsection{Overfitting}

\section{Domänenklassen}

\subsection{Clustern}

\subsection{Regression}

\subsection{Klassifikation}

\subsection{Vorhersage}

\subsection{Robotics}

\subsection{Computer Vision}

\subsection{Optimierungsprobleme}

\cite{AI3}

\section{Neuronale Netzwerktypen}

\subsection{Self-Organizing Map}

\subsection{FeedForward}

\subsection{Hopfield}

\subsection{Boltzmann Machine}

\subsection{Deep Belief Network}

\subsection{Deep Feedforward}

\subsection{NEAT}

\subsection{CPPN}

\subsection{HyperNEAT}

\subsection{Convolutional neural network}

\subsection{Elman Network}

\subsection{Jordan Network}

\subsection{Recurrent Network}

\section{Tensorflow Typen Unterstützung}
