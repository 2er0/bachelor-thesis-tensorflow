
\chapter{Kurzfassung}

NN sind seit Jahren vorhanden
Zeitlich bedingt nicht umsetzbar da technische Voraussetzung nicht erfüllt
werden immer mehr eingesetzt 
Unterstützung im Alltagsleben

Ziel der Arbeit eine Einführung in die Welt der NNN mit Hilfe eines Frameworks


%Maschinelles lernen und tiefe neuronale Netze werden unter anderem in 
%unserem Jahrzehnt sehr häufig eingesetzt um technische Problemstellungen 
%zu lösen, für des eine in vernünftiger Zeit keine Brauchbaren 
%iterative Algorithmen gibt. Zusätzlich werden maschinell 
%lernende System immer häufiger in unserem allgemeinen Alttag 
%eingesetzt um uns zu unterstützen und um von den Benützern zu lernen.
%Ein neuronales Netzwerk kann aber nicht einfach erstellt werden
%und im nächsten Schritt in der Praxis eingesetzt werden. Dies würde 
%zu erheblichen Problemen führen. Diese Netzwerke müssen trainiert werden
%sowie getestet. 


%Im Rahmen dieser Bachelorarbeit werden die wichtigsten theoretischen Konzepte zu 
%maschinellem lernen und tiefe neuronale Netze theoretisch zu vergleichen und empirisch zu überprüfen. 
%Dazu wird das TensorFlow-Bibliothek als Beispiel verwendet und analysiert. Aus dieser Bibliothek werden 
%die benötigten Teile heraus genommen und in einem Python-Script zusammen gefügt um in Bildern mit Gesichtern 
%gewisse Züge zu erkennen und zu Klassifizieren. Durch die Analysephasen werden theoretische und berechnete Annahmen 
%untermauert oder in Frage gestellt. Folgende Schlussfolgerungen gehen jedoch nach Auswertung 
%der Ergebnisse über die theoretischen Annahmen hinaus: Ist das TensorFlow-System 
%in der Lage, Muster aus unterschiedlichen Datentypen, wie zum Beispiel Bilder, 
%Videos oder Videostreams, zu erkennen und von diesen selbst zu lernen?

%Die Bachelorarbeit ist sowohl für Studierende im Studium Software Engineering sowie 
%Informatik als auch für Lehrende in diesen Bereichen interessant.




%An dieser Stelle steht eine Zusammenfassung der Arbeit, Umfang
%max.\ 1 Seite. Im Unterschied zu anderen Kapiteln ist die
%Kurzfassung (und das Abstract) üblicherweise nicht in Abschnitte
%und Unterabschnitte gegliedert. 
%Auch Fußnoten sind hier falsch am Platz.
%
%Kurzfassungen werden übrigens häufig -- zusammen mit Autor und Titel
%der Arbeit -- %
%in Literaturdatenbanken aufgenommen. Es ist daher darauf zu
%achten, dass die Information in der Kurzfassung für sich 
%\emph{allein} (\dah ohne weitere Teile der Arbeit) zusammenhängend und
%abgeschlossen ist. Insbesondere werden an dieser Stelle (wie \ua
%auch im \emph{Titel} der Arbeit und im \emph{Abstract})
%normalerweise \emph{keine Literaturverweise} verwendet! Falls
%unbedingt solche benötigt werden -- etwa weil die Arbeit eine
%Weiterentwicklung einer bestimmten, früheren Arbeit darstellt --,
%dann sind \emph{vollständige} Quellenangaben in der Kurzfassung
%selbst notwendig, \zB %
%[\textsc{Zobel} J.: \textit{Writing for Computer Science -- The Art of
%Effective Commu\-nica\-tion}. Springer-Verlag, Singa\-pur, 1997].
%
%Weiters sollte daran gedacht werden, dass bei der Aufnahme in Datenbanken
%Sonderzeichen oder etwa Aufzählungen mit "`Knödellisten"' in der
%Regel verloren gehen. Dasselbe gilt natürlich auch für das 
%\emph{Abstract}.
%
%
%Inhaltlich sollte die Kurzfassung \emph{keine} Auflistung der
%einzelnen Kapitel sein (dafür ist das Einleitungskapitel
%vorgesehen), sondern dem Leser einen kompakten, inhaltlichen
%Überblick über die gesamte Arbeit verschaffen. Der hier verwendete
%Aufbau ist daher zwangsläufig anders als der in der Einleitung.
