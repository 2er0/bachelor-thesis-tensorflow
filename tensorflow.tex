\chapter{TensorFlow}
\label{cha:TensorFlow}

TensorFlow repräsentiert eine Bibliothek für Machine Intelligence. 
Historisch gesehen entstand TensorFlow in der Google Brain Abteilung.
Das Projekt wird als Open Source Projekt weiterentwickelt, wobei das Projekt von Google weiterhin gepflegt wird. 
Das offenlegen des Projekt führt dazu, dass auch Personen außerhalb von Google die Möglichkeit bekommen, die Bibliothek zu verwenden sowie dazu etwas bei zu tragen. \newline

\noindent
Das Hauptkonzept in TensorFlow sind sogenannte Tensoren welche einen Graphen durchlaufen. 
%Diese Tensoren werden während ihrem durch lauf verändert und wieder neu zusammengesetzt. 
Der Graphen selbst stellt damit einen Datenflussgraphen dar, welcher Knoten beinhaltet. 
Diese Knoten bilden nummerische Operationen ab.
Der Informationsaustausch zwischen den Knoten geschieht mit multidimensionalen Arrays den so genannten Tensoren.
TensorFlow bietet wie andere Bibliotheken die Möglichkeit die Berechnungen auf eine Grafikkarte auszulagern.
Zusätzlich sind weite Routinen eingebaut damit das Trainieren verteilt werden kann über mehrere Grafikkarten sowie auf weitere Computer. \newline

\noindent
TensorFlow steht für mehrere Programmiersprachen zur Verfügung welche Offiziell unterstützt werden, wobei es noch mehr durch die Open Source Gemeinschaft unterstützte Sprachen gibt.
Den Hauptbereich stellt die Python API dar, welche auch die vollständigste Implementierung darstellt. 
Der Kern von TensorFlow ist mit C++ und Python implementiert und wurde sehr stark optimiert, um eine sehr gute Performanz zu erzielen.
Die Python API wird im Umfeld von TensorFlow dazu verwendet, um einen Graphen zu erstellen, zu trainieren und zu testen. 
Durch die Verwendung von Python besteht die Möglichkeit sehr schnell Änderungen am Graphen durchzuführen und nicht erst ganze Applikationsstrukturen zu übersetzten damit ein Ergebnis der Änderung ersichtlich wird. 
Dieser Graphen wird nach seiner Trainingsphase exportiert und beinhaltet alle Knoten sowie die dazugehörigen Gewichtungen. 
Die C++ API sowie die Java API und GO API zielen auf eine sehr effiziente Ausführung ab.
Durch die Verwendung des trainierten Graphen kann dieser auch auf mobiles Plattformen eingesetzt werden.

\section{Bibliotheksinhalt}

\subsection{Graphs / Dataflowgraph}

\begin{figure}

\lstset{language=Python}
\begin{lstlisting}
import tensorflow as tf

b = tf.Variable(tf.zeros([100])) 
	# 100-d Vektor, initialisiert mit 0
W = tf.Variable(tf.random_uniform([784,100],-1,1)) 
	# 784x100 Matrix w/rnd vals
x = tf.placeholder(name="x") 
	# Platzhalter für Eingangsdaten
relu = tf.nn.relu(tf.matmul(W, x) + b) 
	# Relu(Wx+b) Aktivierungsfunktion mit impliziter Addition
C = [...] 
	# Kostenfunktion und noch weitere Knoten
s = tf.Session()
for step in xrange(0, 10):
	input = ...construct 100-D input array ... 
		# Erstellen eines 100-d Vektor mit den Eingangsdaten
	result = s.run(C, feed_dict={x: input}) 
		# Graphen mit den Eingangsdaten ausführen
	print step, result 
		# Ausgabe des Berechneten Resultats
\end{lstlisting}

	\caption{TensorFlow Codefragment zur Definition eines Teils des Graphen}
	\label{fig:SimpleFragmentGraphDefinition}
\end{figure}

\begin{figure}

	\centering

\begin{tikzpicture}

	\node[neuron] (x) {x};
	\node[neuron,below=of x] (w) {W};
	
	\node[group,fit={(x) (w)}] (gr1) {};
	
	\node[neuron,right=of x] (MatMul) {MatMul};
	\node[io,below=of MatMul] (b) {b};
	
	\node[group,fit={(x) (MatMul)},right=of x] (gr2) {};
	
	\node[neuron,right=of MatMul] (Add) {Add};
	
	\node[neuron,right=of Add] (ReLU) {ReLU};
	
	\node[neuron,right=of ReLU] (more) {...};
	
	\node[neuron,right=of more] (C) {C};
	
	\draw[conn] (x) -- (MatMul);
	\draw[conn] (w) -- (MatMul);

	\draw[conn] (MatMul) -- (Add);
	\draw[conn] (b) -- (Add);
	
	\draw[conn] (Add) -- (ReLU);
	\draw[conn] (ReLU) -- (more);

	\draw[conn] (more) -- (C);

\end{tikzpicture}

	\caption{Der resultierenden Teilgraph aus dem Codefragment aus Abbildung \ref{fig:SimpleFragmentGraphDefinition} nach dem Beispiel in \cite{wp2015tensorflow}}
	\label{fig:SimpleFragmentGraphPic}
\end{figure}



\subsection{Operation}

\subsection{Tensor}

\subsection{Operationen}

\subsubsection{Konstanten, Zufallswerte}

\subsubsection{Variables}

\subsubsection{Transformationen}

\subsubsection{Mathematik}

\subsubsection{Flusskontrolle}

\subsubsection{Images / FFmpeg}

\subsubsection{Input und Readers}

\subsubsection{Neural Network}

\subsubsection{Running Graphs}

\subsubsection{Training}

\subsection{Probleme}

\subsubsection{NaN Problem}