\chapter{Abstract}

\begin{english} 
Neuronal networks have been reasearched for decades and are growing in commercial usage. 
They make it possibility to create systems that can solve very complex problems, e.g. the translation of pictures to text. \footnote{Google Translate app: \url{https://research.googleblog.com/2015/07/how-google-translate-squeezes-deep.html}} 
Such systems need computational power which was not available in former times. 
So at the start there were only models and today we can do that nearly on our smartphones. 
The development of integrated circuits enabled more research and a more efficient development in that research area. 
But these systems need more computational power for training than our smartphones currently have. 
For the training there is time needed to adapt and recognize patterns, like a swimmer how wants to adapt a new technique to be more efficient. 
In general all these systems get more and more evolved and are reaching a level where they are simultaneously integrated in our daily life. 

\noindent
The present thesis includes an example with real data which was developed to visualize a problem and how it could be solved. 
Additionally the work contains an introduction to the topic of neuronal networks, it's basics and the basic math behind it. 
By applying the basics to the framework \textit{TensorFlow}, it will get more practical and understandable. 
For people who are technically not experienced, frameworks like \textit{Keras} exist. 
\end{english}
