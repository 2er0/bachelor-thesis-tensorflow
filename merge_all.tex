%% File encoding: UTF-8
%% äöüÄÖÜß  <-- keine deutschen Umlaute hier? UTF-faehigen Editor verwenden!

%\documentclass[praktikum,german]{hgbthesis}
% Zulässige Class Options: 
%   Typ der Arbeit: diplom, master (default), bachelor, praktikum 
%   Hauptsprache: german (default), english
%%------------------------------------------------------------

\documentclass[10pt,a4paper]{article}
%\usepackage[utf8]{inputenc}

\RequirePackage[utf8]{inputenc}		% remove when using lualatex oder xelatex!

%\graphicspath{{images/}}    % name of directory containing the images
%\logofile{logo}							% name of logo-PDF in images/ (or use \logofile{} for no logo)
%\bibliography{literatur}  	% name of the BibTeX (.bib) file

\usepackage{pdfpages}
\usepackage{listings}
\usepackage{graphicx}
\usepackage{tikz}
\usepackage{pgfplots}
\usepackage{subfig}
%\usepackage{titletoc}
\usepackage{minitoc}
\usetikzlibrary{
  arrows.meta, % for Straight Barb arrow tip
  fit, % to fit the group box around the central neurons
  positioning, % for relative positioning of the neurons
}

\tikzset{
  neuron/.style={ % style for each neuron
    circle,draw,thick, % drawn as a thick circle
    inner sep=5pt, % no built-in padding between the text and the circle shape
    minimum size=1.5em, % make each neuron the same size regardless of the text inside
    node distance=1ex and 3em, % spacing between neurons (y and x)
  },
  group/.style={ % style for the groups of neurons
    rectangle,%,draw,thick, % drawn as a thick rectangle
    inner sep=0pt, % no padding between the node contents and the rectangle shape
  },
  io/.style={ % style for the inputs/outputs
    neuron, % inherit the neuron style
    fill=gray!15, % add a fill color
  },
  conn/.style={ % style for the connections
    -{Straight Barb[angle=60:2pt 3]}, % simple barbed arrow tip
    thick, % draw in a thick weight to match other drawing elements
  },
}

\usepackage{array,booktabs,ragged2e}
\newcolumntype{R}[1]{>{\RaggedLeft\arraybackslash}p{#1}}

%\makeatletter
%\@addtoreset{chapter}{part}
%\makeatother  

%%%----------------------------------------------------------
%\begin{document}
%%%----------------------------------------------------------

% Einträge für ALLE Arbeiten: --------------------------------
%\title{Routing in der Logistik}
%\author{David Baumgartner}
%\studiengang{Software Engineering}
%\studienort{Hagenberg}
%\abgabedatum{2017}{01}{14}	% {YYYY}{MM}{DD}

%%% zusätzlich für eine Bachelorarbeit: ---------------------
%\nummer{1410307050-A}   % XX...X = Stud-ID, z.B. 0310238045-A  
                        % (A = 1. Bachelorarbeit)
%\semester{Sommersemester 2017} 
%\betreuer{Stephan Dreiseitl, FH-Prof. PD DI Dr.  \\ DI Dominik Angerer} % oder \betreuerin{..}

%%% zusätzlich für einen Praktikumsbericht: -----------------
%\nummer{1410307050-B}   % XX...X = Stud-ID, z.B. 0310238045-B  
                        % (B = 2. Bachelorarbeit)
%\betreuer{Mag.~Pjotr I.~Czar\\Creative Director}  % \betreuerin{..}
%\firma{%
%   ITPRO - Consulting \& Software GmbH\\
%   4020 Linz, Buchnerplatz 1
%}
%\firmenTel{+43 732 61 51 41}
%\firmenUrl{www.itpro.at}

%\strictlicense  % erzeugt restriktive Lizenzformel

%%%----------------------------------------------------------
\begin{document}
\includepdf[pages={-}]{deckblaetter/Titelblattvorlage_Gesamtarbeit_neu0216.pdf}
\includepdf[pages={-}]{_DaBa.pdf}
\includepdf[pages={1-34}]{_DaBa_Teil2.pdf}
\end{document}
