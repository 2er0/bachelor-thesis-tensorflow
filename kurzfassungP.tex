\chapter{Kurzfassung}

Routing ist allgegenwärtig in Form von Navigationssystemen wie zum Beispiel Google Maps. 
Diese Navigationslösungen besitzen nur die Fähigkeit einfache \textit{point-to-point} Routen zu finden. 
Dabei können Zwischenstopps vorhanden sein, aber mit einer fixen Reihenfolge. 
Routing und im speziellen Transportlogistik beinhalten mehr als nur Navigationsoptimierung. 
So müssen im Falle von UPS Paketauslieferungen optimiert werden, um Einsparungen zu ermöglichen. 

\noindent
Im Rahmen dieser Arbeit wurden die Problemstellungen wie \textit{Traveling Salesman Problem} und weitere erarbeitet und erklärt. 
Des Weiteren wurde auf mögliche Lösungsansätze für solche Probleme eingegangen. 
Die Abläufe solcher Lösungsmöglichkeiten stellen einen weiteren Inhalt dar. 
Im Speziellen wird genauer auf den \textit{Savings-Algorithmus} eingegangen und eine Beispielimplementierung mit Zeitfenster durchgeführt. 