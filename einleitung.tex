\chapter{Einleitung}
\label{cha:Einleitung}

\section{Motivation}

%Zeitlich immer mehr Relevant
Kaum ein Gebiet existiert so lange und erlebte in den letzten 20 Jahren so einen Sprung nach Vorne in den Punkten relevant und Weiterentwicklung. 


\section{Problemstellung}

\section{Zielsetzung}

Einführung in das Thema NNN

%\section{Allgemeines und Motivation}

%Dieses Dokument ist als vorwiegend technische Starthilfe für das
%Erstellen einer Masterarbeit (oder Bachelorarbeit) mit \latex
%gedacht und ist die Weiterentwicklung einer früheren
%Vorlage\footnote{Nicht mehr verfügbar.} für das Arbeiten mit
%Microsoft \emph{Word}. Während ursprünglich daran gedacht war, die
%bestehende Vorlage einfach in \latex zu übernehmen, wurde rasch
%klar, dass allein aufgrund der großen Unterschiede zum Arbeiten
%mit \emph{Word} ein gänzlich anderer Ansatz notwendig wurde. Dazu
%kamen zahlreiche Erfahrungen mit Diplomarbeiten in den
%niachfolgenden Jahren, die zu einigen zusätzlichen Hinweisen Anlass gaben.

%Das vorliegende Dokument dient einem zweifachen Zweck: 
%\emph{erstens} als Erläuterung und Anleitung, \emph{zweitens} als
%direkter Ausgangspunkt für die eigene Arbeit. Angenommen wird,
%dass der Leser bereits über elementare Kenntnisse im Umgang mit
%\latex verfügt. In diesem Fall sollte -- eine einwandfreie
%Installation der Software vorausgesetzt -- der Arbeit nichts mehr
%im Wege stehen. Auch sonst ist der Start mit \latex\ nicht
%schwierig, da viele hilfreiche Informationen auf den zugehörigen
%Webseiten zu finden sind (s.\ Kap.~\ref{cha:Einleitung}).


%\section{Ziel der Arbeit}