\chapter{Einleitung}
\label{cha:Einleitung}

\section{Motivation}

Kaum ein anderes Gebiet der Informationstechnologie existiert so lange wie maschinelles Lernen und erlebte in den letzten 20 Jahren einen Aufschwung in den Punkten Relevanz, Weiterentwicklung und Alltagstauglichkeit. 
Ein Umstand dafür ist unter anderem die technische Voraussetzung große Systeme entwickeln zu können und diese auch der Menschheit zugänglich zu machen. 
Im Speziellen wurde hierbei der Fokus auf neuronale Netzwerke gelegt, welche noch weiter erforscht werden, aber auch schon im Einsatz sind. \newline

\noindent
Die großen Datenmengen, die in den letzten Jahren zu verarbeiten und zu analysieren waren, stellen ein Problem dar. 
So befindet sich kein Mensch in der Lage, effizient Millionen an Datensätzen zu analysieren und Zusammenhänge darin zu finden und dies in adäquater Zeit zu tun. \newline

\noindent
Im Zuge dieser Arbeit sollen die Grundlagen im Bereich des maschinellen Lernens, im Speziellen mit neuronalen Netzwerken näher gebracht werden. 
Dabei soll gezeigt werden, wie weit sich diese in der Praxis einsetzen lassen. 

\section{Problemstellung}

Die grundlegende Problemstellung lässt sich mit folgender Frage beschreiben: \newline

Wie weit ist TensorFlow als Bibliothek im Gebiet des maschinellen Lernens in praktischen Fällen einsatzfähig? \newline

\noindent
In dieser Frage stecken mehrere nichttriviale Punkte. 

\noindent
Ein Punkt stellt das Gebiet des maschinellen Lernens generell dar, sowie die praktische Umsetzung von Problemstellungen. 
Im Speziellen bieten neuronale Netzwerke neue Möglichkeiten, Probleme in der Informationstechnologie zu lösen, sowie Vorgänge in der Natur besser zu verstehen. \newline

\noindent
Zum anderen die Frage, was ist TensorFlow, wofür steht es und wofür kann dies eingesetzt werden?  
Diese Bibliothek bietet eine gute Unterstützung sich dem Gebiet des maschinellen Lernens zu nähern, aber auch die Möglichkeit, dies in praktischen Fällen einzusetzen. 

\section{Zielsetzung}

Die Arbeit soll das Thema maschinelles Lernen - neuronale Netzwerke so beleuchten, dass es möglich ist, diese mit den Grundlagen zu verstehen und zu erlernen. 
Des Weiteren wird die Bibliothek TensorFlow näher gebracht, welche eine gute Unterstützung bietet, eine Problemstellung zu lösen und diese Lösung direkt in einer Applikation einzusetzen. \newline

\noindent
Die Betrachtung der benötigten Punkte und Gebiete erfolgt auf breiter Front, am Ende sollte jeder Leser ein Verständnis dafür haben. 
Außerdem sollte er in der Lage sein können, den Umfang einer solchen Aufgabenstellung festzustellen. 
Als Konsequenz aus dieser Betrachtungsweise können allerdings nicht alle Punkte in ihrer ganzen Tiefe erfasst werden. 
Hier wäre es erforderlich, noch weitere Literatur einzubeziehen und diese zu studieren. \newline

\noindent
Am Ende der Arbeit wird eine mögliche Lösung für ein praktisches Beispiel entwickelt und näher erklärt.
Dieses wird als praktische Repräsentation verwendet, um den Umfang von TensorFlow noch besser zu verstehen. \newline

\noindent
Der Anspruch, dass mit dieser Technik der Datenanalyse jegliche Problemstellung gelöst werden kann, ist ausdrücklich kein Ziel dieser Arbeit. 
Des Weiteren wird teilweise nicht tiefer auf Themen eingegangen, da dies den Rahmen dieser Arbeit übersteigen würde. 
Diese Arbeit sollte aber als möglicher Startpunkt, für einen Einstieg in die Welt des maschinellen Lernens - neuronale Netzwerke, dienen. 