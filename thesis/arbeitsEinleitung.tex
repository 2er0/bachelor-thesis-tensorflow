\chapter{Einleitung}
\label{cha:Einleitung}

\section{Motivation}

Der Gütertransport existiert seit Waren transportiert werden. 
Nicht nur der Personenverkehr nimmt immer mehr zu, sondern auch der Gütertransport. 
Es werden immer mehr Güter weltweit erzeugt, dafür müssen zum Abnehmer meist weite Strecken zurückgelegt werden. 
Dies führt dazu, dass Straßen ausgebaut und Hauptverkehrsrouten adaptiert werden müssen. 
Vom Jahr 2014 auf 2015 nahm zum Beispiel das Güteraufkommen in Deutschland um $1,9\,\%$ zu.\footnote{Bundesverband Güterkraftverkehr; \url{http://bgl-ev.de}} 
Damit die Kosten konstanter bleiben, müssen Transportrouten optimierter erledigt werden.
Dadurch werden Kosteneinsparungen erzielt. 
So wird auch der Umwelt etwas Gutes getan. 

\section{Problemstellung}

Ein Problem in der Logistik stellt die Optimierung von Routen dar. 
Optimierungen von Routen bieten eine Möglichkeit, um effektiver Kunden zu erreichen. 
Dabei kann nicht nur Zeit gespart werden, sondern auch Fahrzeuge besser ausgenützt werden. 
Zum Beispiel erspart sich UPS durch intelligente Optimierung Millionen an Kilometern. 
Der Paketdienstleister UPS entwickelte sich unter anderem auch zu einem Technikunternehmen mit tausenden Servern. 
Diese lösen und analysieren ununterbrochen Routen, damit am Morgen sofort optimierte Touren zur Verfügung stehen. 
Das dafür entwickelte System nennt sich \textit{ORION} und analysiert in Realzeit anhand der Verkehrslage. 
Dafür wurde seit 2003 an diesem System gearbeitet und erst im Jahr 2012 mit dem Betatesting begonnen.\footnote{UPS ORION; \url{https://www.pressroom.ups.com}}
Solche Probleme betreffen nicht nur Auslieferungsdienste wie UPS, sondern auch lokale Produzenten mit eigener Auslieferung. 
Diese sind meistens zusätzlich auf Zeitfenster bei den Kunden angewiesen, um einen Mehrwert für die Kunden zu bieten. 
Mit diesen Zeitfenstern wird die Problemstellung um einiges komplexer und schwerer lösbar. 

\section{Zielsetzung}

Diese Arbeit soll die Grundlagen für Optimierungen in der Logistik näher bringen und einen Einstieg darstellen. 
Hierbei wird auch auf mögliche Kostenberechnungen und Kostenmatrizen eingegangen und Probleme aufgezeigt. 
Am Ende wird das Gebiet der Zeitfenster anhand eines Beispiels aufgearbeitet und eine mögliche Lösung präsentiert. 