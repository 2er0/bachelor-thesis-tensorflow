\chapter{Kurzfassung}

Neuronale Netzwerke werden seit Jahrzehnten erforscht und auch seit geraumer Zeit eingesetzt. 
Solche Netzwerke ermöglichen es, komplexe Systeme für logisch sehr komplexe Aufgaben zu entwickeln. 
Zum Beispiel werden bei der Übersetzung von Texten in Bildern neuronale Netzwerke eingesetzt. 
Diese Ausführung geschieht dabei zum Teil auf den Smartphones lokal. 
Eine lokale Ausführung war dabei nicht immer möglich, da die dafür benötigten Ressourcen nicht vorhanden waren. 
Zu Beginn der Erforschung von neuronalen Netzwerken konnten nur Modelle erstellt werden. 
Modelle wurden deshalb herangezogen, da die technischen Voraussetzungen noch nicht gegeben waren. 
Mit der Entwicklung von integrierten Schaltkreisen und immer leistungsfähigeren Recheneinheiten konnte diese Grundlage geschaffen werden. 
Dabei existieren noch Probleme, denn ein maschinell lernendes System muss trainiert werden. 
Ein solches Training kann zum Beispiel mit einem Schwimmer verglichen werden, der einen neuen Technikablauf integrieren möchte. 
So ein Vorgang benötigt viel Zeit bis der Ablauf adaptiert wird und im Anschluss fast vollautomatisch abläuft. 
Ähnlich geht es den neuronalen Netzwerken, welche anhand des Ergebnisses angepasst werden müssen. 
Der Einzug von maschinell lernenden Systemen in die Zivilgesellschaft ist dank ihres Erfolgs praktisch unumgänglich. 

\noindent
Im Rahmen dieser Bachelorarbeit wurde ein Beispiel anhand realer Daten umgesetzt und implementiert. 
An diesem Beispiel werden einige Eigenheiten solcher Systeme erklärt und beschrieben. 
Zusätzlich beinhaltet diese Arbeit auch eine Einführung in das Gebiet der neuronalen Netzwerke. 
Diese Grundlagen werden mit Hilfe der Bibliothek \textit{TensorFlow} vervollständigt. 
\textit{TensorFlow} bietet eine guten Einstieg, damit die Grundlagen besser und praktischer verstanden werden. 
Technisch nicht so versierte Personen sollten nach dem Erlernen der Grundlagen deshalb eine Abstraktion verwenden, wie zum Beispiel \textit{Keras}. 
